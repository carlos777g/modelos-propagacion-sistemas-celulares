\section{\Large Conclusión}

A lo largo de esta práctica, hemos analizado diferentes modelos de propagación de ondas en entornos urbanos, lo que nos permitió comprender mejor su comportamiento en condiciones reales. Los resultados obtenidos muestran claramente cómo factores como los edificios, la disposición de las calles y las reflexiones afectan significativamente la calidad de la señal.

El modelo de espacio libre, resulta claro para entornos urbanos complejos. Por otro lado, el modelo Walfish-Ikegami, al incorporar obstáculos como edificios y la geometría de las calles, ofrece predicciones más realistas, además pudimos observar cómo la presencia o ausencia de línea de vista directa puede crear diferencias en la potencia recibida.

Esta práctica no solo nos permitió comparar diferentes enfoques teóricos, sino que también nos mostró la importancia de seleccionar el modelo adecuado según las características específicas del entorno.