% Esenciales:
\usepackage[a4paper, margin=2.5cm]{geometry} % Hoja A4 con margen de 2.5cm
\usepackage{tikz,pgfplots}
\usepackage{graphicx}
\usepackage{float}
\usepackage{amsmath}

% Tipo de letra
% Arial:
\renewcommand{\familydefault}{\sfdefault}
% Comentar Arial y descomentar las siguientes para Helvetica:
% \usepackage{helvet}
% \renewcommand{\familydefault}{\sfdefault}

% Para citas bibliograficas:
\usepackage[backend=bibtex, style=numeric]{biblatex}
\addbibresource{bib/referencias.bib}

% Índice clickeable
% \usepackage[colorlinks=true, linkcolor=black]{hyperref}
\usepackage[
  colorlinks=true,
  linkcolor=black,      % Títulos, referencias internas, índice
  citecolor=black,      % Citas [1], [2], etc.
  urlcolor=blue         % Enlaces web
]{hyperref}

\usepackage[utf8]{inputenc}
\usepackage[T1]{fontenc}
\usepackage{listings}          % Para código
\usepackage{xcolor}            % Para colores


\usepackage{titlesec}   % Para modificar títulos
\usepackage{tocloft}    % Para personalizar el índice
% --- Ocultar números en títulos (pero mantener en índice) ---
\titleformat{\section}
  {\color{cyan!45!black}\normalfont\Large\bfseries}{}{0em}{}

\titleformat{\subsection}
{\color{black}\normalfont\large\bfseries}{}{0em}{}  % Sin número visible y con color
  

% --- Asegurar que el índice muestre los números ---
\renewcommand{\cftsecaftersnum}{.}       % Punto después del número (opcional)
\renewcommand{\cftsubsecaftersnum}{.}    % Ej: "1.2." en el índice